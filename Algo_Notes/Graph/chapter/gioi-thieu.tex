\chapter{Giới thiệu}
\minitoc 

Bài viết này được biên soạn với mục tiêu giúp tác giả hệ thống hoá và vận dụng các kiến thức thuộc chuyên đề \textit{Lý thuyết đồ thị}, từ đó áp dụng hiệu quả trong \textit{Competitive Programming} (Lập trình thi đấu).

\section{Các nguồn tài nguyên}
\begin{quote}
    \begin{itemize}
        \item C{\tt/}C++: \url{https://github.com/GrootTheDeveloper/OLP-ICPC/tree/master/2025/C%2B%2B}
        \item \cite{CP10}. \textit{CP10. Competitive Programming} \url{https://drive.google.com/drive/folders/1MTEVHT-7nBnMJ7C9LgyAR_pEVSE3FlKz?fbclid=IwAR3TovIj2rKCRe1a4oZxW-LQCoEoVkipVAvCzwrr0nJ6GzcAd47P6LO1Rwc}
	
        \item \cite{cp-algorithms}. \textit{Algorithms for Competitive Programming} \url{https://cp-algorithms.com}

        \item \cite{VNOI-WIKI}. \textit{Thư viện VNOI} \url{https://wiki.vnoi.info}
    \end{itemize}
\end{quote}

\section{Tài khoản trên các Online Judge}
\begin{quote}
    \begin{itemize}
        \item Codeforces: \url{https://codeforces.com/profile/vuivethoima}
        \item VNOI: \url{oj.vnoi.info/user/Groot}
        \item IUHCoder: \url{oj.iuhcoder.com/user/ankhang2111}
        \item MarisaOJ: \url{https://marisaoj.com/user/grootsiuvip/submissions}
        \item CSES: \url{https://cses.fi/user/212174}
        \item UMTOJ: \url{sot.umtoj.edu.vn/user/grootsiuvip}
        \item SPOJ: \url{www.spoj.com/users/grootsiuvip/}
        \item POJ: \url{http://poj.org/userstatus?user_id=vuivethoima}
        \item AtCoder: \url{https://atcoder.jp/users/grootsiuvip}
        \item OnlineJudge.org: \url{vuivethoima}
    \end{itemize}
\end{quote}

\section{Một vài lưu ý}
\subsection*{[25/8/2025]: }
Chuyên đề này được viết bởi hai “tác giả” (bắt đầu xuất hiện từ phần \textbf{DSU}):

\begin{itemize}
    \item \textbf{vuivethoima} – tác giả chính, chịu trách nhiệm biên soạn nội dung.
    \item \textbf{Groot} – một thằng chuyên chọc ngoáy, đặt những câu hỏi nghe thì rất ngu ngơ nhưng lại gợi mở những góc khuất của bài toán mà thường ít ai để ý (chắc vậy?).
\end{itemize}

Nói cho sang thì là ``cộng tác'', nhưng thực chất đây là quá trình DPAK tự viết, rồi tự hỏi, rồi tự tranh luận. Hai ``nhân vật'' trong đầu thay phiên nhau đóng vai \textit{tác giả} và \textit{độc giả khó tính}. Và thế là hình thành nên chuyên đề này.
