\chapter{TÌM KIẾM NHỊ PHÂN}

\minitoc

\section{Lý thuyết}

\subsection{Khái niệm}
Tìm kiếm nhị phân là \textbf{thuật toán tìm kiếm trên mảng đã được sắp xếp} (theo thứ tự tăng dần hoặc giảm dần).  
Ý tưởng chính: \textbf{chia đôi khoảng tìm kiếm} sau mỗi bước, từ đó giảm số lượng phần tử cần xét xuống một nửa.

\begin{itemize}
    \item Nếu giá trị cần tìm \textbf{nhỏ hơn} giá trị giữa $\rightarrow$ tìm ở \textbf{nửa trái}.
    \item Nếu giá trị cần tìm \textbf{lớn hơn} giá trị giữa $\rightarrow$ tìm ở \textbf{nửa phải}.
    \item Nếu bằng giá trị giữa $\rightarrow$ \textbf{tìm thấy} phần tử.
\end{itemize}

\textbf{Điều kiện bắt buộc:} Mảng phải được \textbf{sắp xếp trước} khi áp dụng tìm kiếm nhị phân.

\subsection{Độ phức tạp}

\begin{center}
\begin{tabular}{|c|c|}
\hline
\textbf{Bước phân tích} & \textbf{Giá trị} \\
\hline
Thời gian & $O(\log_2 n)$ \\
\hline
Không gian & $O(1)$ (phiên bản lặp) hoặc $O(\log n)$ (phiên bản đệ quy) \\
\hline
\end{tabular}
\end{center}

Vì mỗi lần chia đôi số phần tử, nên số bước lặp tỷ lệ với $\log_2 n$.


\subsection{Ý tưởng thuật toán}
Giả sử mảng $a$ có $n$ phần tử, tìm số $x$:

\begin{enumerate}
    \item \textbf{Khởi tạo:} 
    \begin{itemize}
        \item \texttt{left = 0}, \texttt{right = n-1}.
    \end{itemize}
    
    \item \textbf{Trong khi} \texttt{left <= right}:
    \begin{itemize}
        \item \texttt{mid = (left + right) / 2}
        \item Nếu \texttt{a[mid] == x} $\rightarrow$ tìm thấy $\rightarrow$ trả về vị trí.
        \item Nếu \texttt{a[mid] < x} $\rightarrow$ tìm ở \textbf{nửa phải} $\rightarrow$ \texttt{left = mid + 1}.
        \item Nếu \texttt{a[mid] > x} $\rightarrow$ tìm ở \textbf{nửa trái} $\rightarrow$ \texttt{right = mid - 1}.
    \end{itemize}
    
    \item Nếu vòng lặp kết thúc mà chưa tìm thấy $\rightarrow$ $x$ không tồn tại trong mảng.
\end{enumerate}

\section{Bài tập tìm kiếm nhị phân}

\begin{baitap}{Tìm kiếm nhị phân}{https://marisaoj.com/problem/515}
Cho mảng $A$ được sắp xếp tăng dần gồm $n$ số nguyên khác nhau.  
Có $q$ truy vấn, mỗi truy vấn là một số nguyên $x$.  
Hãy xác định xem giá trị $x$ nằm ở vị trí nào trong mảng $A$.

\textbf{Input}
\begin{itemize}
    \item Dòng đầu tiên gồm hai số nguyên $n, q$ --- số phần tử của mảng và số lượng truy vấn. 
    \item Dòng thứ hai gồm $n$ số nguyên $A_i$ $(1 \leq i \leq n)$.
    \item $q$ dòng tiếp theo, mỗi dòng gồm một số nguyên $x$, biểu thị một truy vấn.
\end{itemize}

\textbf{Output}
Với mỗi truy vấn, in ra vị trí của số nguyên $x$ trong mảng $A$.  
Đảm bảo rằng $x$ luôn tồn tại trong mảng $A$.

\textbf{Giới hạn}
\begin{itemize}
    \item $1 \leq n, q \leq 10^5$
    \item $1 \leq A_i, x \leq 10^9$
\end{itemize}

\begin{simple_example}
5 3 & 2 \\
1 4 6 8 10 & 5 \\
4 & 1 \\
10 & \\
1 & \\
\end{simple_example}

Mảng được đánh chỉ số bắt đầu từ 1. \\

\end{baitap}

\begin{baitap}{Tìm kiếm nhị phân 2}{https://marisaoj.com/problem/76}
Cho mảng $A$ \textbf{không giảm} gồm $n$ số nguyên. Có $q$ truy vấn, mỗi truy vấn cho một số nguyên $x$.  
Hãy tìm chỉ số nhỏ nhất $i$ sao cho $A_i = x$. Nếu không tồn tại, in ra $-1$.

\textbf{Input}
\begin{itemize}
    \item Dòng đầu tiên gồm hai số nguyên $n, q$.
    \item Dòng thứ hai gồm $n$ số nguyên $A_i$ $(1 \le i \le n)$.
    \item $q$ dòng tiếp theo, mỗi dòng gồm một số nguyên $x$ (một truy vấn).
\end{itemize}

\textbf{Output}
In ra $q$ dòng, dòng thứ $i$ là kết quả cho truy vấn thứ $i$:  
chỉ số nhỏ nhất $i$ sao cho $A_i = x$, hoặc $-1$ nếu không tìm thấy.

\textbf{Giới hạn}
\begin{itemize}
    \item $1 \le n, q \le 10^5$
    \item $1 \le A_i, x \le 10^9$
\end{itemize}

\begin{simple_example}
5 3 & 1 \\     
1 2 3 3 9 & 5 \\ 
1 & 3 \\        
9 & \\     
3 & \\         
\end{simple_example}

\end{baitap}

\begin{baitap}{Tìm kiếm nhị phân 3}{https://marisaoj.com/problem/77}
Cho mảng $A$ \textbf{không giảm} gồm $n$ số nguyên. Có $q$ truy vấn, mỗi truy vấn cho một số nguyên $x$.  
Hãy tìm chỉ số lớn nhất $i$ sao cho $A_i \leq x$. Nếu không tồn tại, in ra $-1$.

\textbf{Input}
\begin{itemize}
    \item Dòng đầu tiên gồm hai số nguyên $n, q$.
    \item Dòng thứ hai gồm $n$ số nguyên $A_i$ $(1 \le i \le n)$.
    \item $q$ dòng tiếp theo, mỗi dòng gồm một số nguyên $x$ (một truy vấn).
\end{itemize}

\textbf{Output}
In ra $q$ dòng, dòng thứ $i$ là kết quả cho truy vấn thứ $i$:  
chỉ số lớn nhất $i$ sao cho $A_i \leq x$, hoặc $-1$ nếu không tìm thấy.

\textbf{Giới hạn}
\begin{itemize}
    \item $1 \le n, q \le 10^5$
    \item $1 \le A_i, x \le 10^9$
\end{itemize}

\begin{simple_example}
5 3 & 1 \\        
1 2 3 3 9 & 4 \\ 
1 & 4 \\   
5 & \\          
3 & \\  
\end{simple_example}

\end{baitap}

\begin{baitap}{Mảng con lớn}{https://marisaoj.com/problem/78}
Cho mảng $A$ gồm $n$ số nguyên dương. Hãy đếm số lượng dãy con liên tiếp có tổng không nhỏ hơn $k$.

\textbf{Input}
\begin{itemize}
    \item Dòng đầu tiên gồm hai số nguyên $n, k$.
    \item Dòng thứ hai gồm $n$ số nguyên $A_i$ ngăn cách bởi dấu cách.
\end{itemize}

\textbf{Output}
In ra một số nguyên là số lượng dãy con liên tiếp có tổng không nhỏ hơn $k$.

\textbf{Giới hạn}
\begin{itemize}
    \item $1 \le n \le 10^5$
    \item $1 \le A_i, k \le 10^9$
\end{itemize}

\begin{simple_example}
5\ 6 & 6 \\
1\ 2\ 1\ 4\ 5 & \\
\end{simple_example}
\end{baitap}

\begin{baitap}{Truy vấn đếm}{https://marisaoj.com/problem/79}
Cho mảng $A$ gồm $n$ số nguyên. Có $q$ truy vấn dạng $(l, r, x)$, hãy đếm số lượng giá trị $x$ trong đoạn $A_l, A_{l+1}, \ldots, A_r$.

\textbf{Input}
\begin{itemize}
    \item Dòng đầu tiên gồm hai số nguyên $n, q$.
    \item Dòng thứ hai gồm $n$ số nguyên $A_i$ cách nhau bởi dấu cách.
    \item $q$ dòng tiếp theo, mỗi dòng gồm ba số nguyên $l, r, k$ cách nhau bởi dấu cách.
\end{itemize}

\textbf{Output}
In ra $q$ dòng, dòng thứ $i$ là kết quả của truy vấn thứ $i$.

\textbf{Giới hạn}
\begin{itemize}
    \item $1 \leq n, q, k, A_i \leq 10^5$
    \item $1 \leq l \leq r \leq n$
\end{itemize}

\begin{simple_example}
7 3 & 2 \\
1 2 1 4 2 4 2 & 3 \\
1 4 1 & 1 \\
2 7 2 & \\
3 5 4 & \\
\end{simple_example}
\end{baitap}

\begin{baitap}{Đếm cặp}{https://marisaoj.com/problem/80}
Cho mảng $A$ gồm $n$ số nguyên. Hãy đếm số cặp $(i,j)$ với $i<j$ sao cho $l \le A_i + A_j \le r$.

\textbf{Input}
\begin{itemize}
    \item Dòng đầu tiên gồm ba số nguyên $n, l, r$.
    \item Dòng thứ hai gồm $n$ số nguyên $A_i$ cách nhau bởi dấu cách.
\end{itemize}

\textbf{Output}
In ra một số nguyên là số cặp $(i,j)$ thỏa mãn $i<j$ và $l \le A_i + A_j \le r$.

\textbf{Giới hạn}
\begin{itemize}
    \item $1 \le n \le 10^5$
    \item $1 \le A_i, l, r \le 10^9$
\end{itemize}

\begin{simple_example}
3\ 2\ 4 & 2 \\
1\ 2\ 3 & \\
\end{simple_example}
\end{baitap}


\begin{baitap}{Viên kẹo thứ k}{https://marisaoj.com/problem/81}
Có $n$ loại kẹo. Loại $i$ có $a_i$ viên và mỗi viên nặng $w_i$. Không có hai loại kẹo có cùng khối lượng. Tất cả kẹo được đổ lên bàn và sắp xếp thành một hàng theo \textbf{khối lượng không giảm}. Với mỗi truy vấn $k$, hãy cho biết viên kẹo thứ $k$ trên bàn nặng bao nhiêu.

\textbf{Input}
\begin{itemize}
    \item Dòng đầu tiên gồm hai số nguyên $n, q$.
    \item $n$ dòng tiếp theo, dòng thứ $i$ gồm hai số nguyên $a_i, w_i$.
    \item $q$ dòng tiếp theo, mỗi dòng gồm một số nguyên $k$.
\end{itemize}

\textbf{Output}
In ra $q$ dòng, dòng thứ $i$ là khối lượng của viên kẹo thứ $k$ ở truy vấn thứ $i$.

\textbf{Giới hạn}
\begin{itemize}
    \item $1 \le n, q \le 10^5$
    \item $1 \le a_i, w_i \le 10^9$
    \item $1 \le k \le \sum_{i=1}^{n} a_i \le 10^{14}$
\end{itemize}

\begin{simple_example}
3\ 3 & 1 \\
2\ 2 & 2 \\
1\ 1 & 3 \\
3\ 3 & \\
1 & \\
3 & \\
5 & \\
\end{simple_example}
\end{baitap}

\begin{baitap}{Cạnh tam giác}{https://marisaoj.com/problem/85}
Cho mảng $A$ gồm $n$ phần tử nguyên. Hãy đếm số lượng bộ $3$ số $(i, j, k)$ với $i<j<k$ sao cho $A_i, A_j, A_k$ có thể tạo thành một bộ cạnh của tam giác.

\textbf{Input}
\begin{itemize}
    \item Dòng đầu tiên gồm một số nguyên $n$.
    \item Dòng thứ hai gồm $n$ số nguyên $A_i$.
\end{itemize}

\textbf{Output}
In ra một số nguyên là số bộ $(i, j, k)$ thỏa mãn điều kiện đề bài.

\textbf{Giới hạn}
\begin{itemize}
    \item $1 \leq n \leq 1500$
    \item $1 \leq A_i \leq 10^9$
\end{itemize}

\begin{simple_example}
4 & 1 \\
1\ 2\ 3\ 4 & \\
\end{simple_example}
\end{baitap}

\begin{baitap}{Số Hamming}{https://marisaoj.com/problem/82}
Số Hamming là các số nguyên dương chỉ có các ước nguyên tố là $2$, $3$, $5$ (nó không chia hết cho số nguyên tố nào ngoài $2$, $3$, $5$).

Cho $q$ truy vấn, mỗi truy vấn là một số nguyên $m$. Viết tất cả các số Hamming theo thứ tự tăng dần và tìm vị trí của số $m$ trong dãy (dãy được đánh số từ $1$).

\textbf{Input}
\begin{itemize}
    \item Dòng đầu tiên gồm số nguyên $q$.
    \item $q$ dòng tiếp theo, mỗi dòng gồm một số nguyên $m$.
\end{itemize}

\textbf{Output}
In ra $q$ dòng, dòng thứ $i$ là kết quả của truy vấn thứ $i$.  
Nếu $m$ không phải số Hamming, in ra \texttt{-1}.

\textbf{Giới hạn}
\begin{itemize}
    \item $1 \leq q \leq 10^5$
    \item $1 \leq m \leq 10^{18}$
\end{itemize}

\begin{simple_example}
5 & 1 \\
1 & 2 \\
2 & 3 \\
3 & 4 \\
4 & -1 \\
7 & \\
\end{simple_example}
\end{baitap}

\begin{baitap}{Số nguyên liên tiếp}{https://marisaoj.com/problem/197}
Cho mảng $A$ gồm $n$ số nguyên. Hãy tìm số lượng phần tử ít nhất cần thay đổi để mảng $A$ chỉ gồm các số nguyên liên tiếp.

\textbf{Input}
\begin{itemize}
    \item Dòng đầu tiên gồm một số nguyên $n$.
    \item Dòng thứ hai gồm $n$ số nguyên $A_i$.
\end{itemize}

\textbf{Output}
In ra một số nguyên là số lượng phần tử ít nhất phải thay.

\textbf{Giới hạn}
\begin{itemize}
    \item $1 \le n \le 10^5$
    \item $1 \le A_i \le 10^9$
\end{itemize}

\begin{simple_example}
3 & 1 \\
4\ 10\ 5 & \\
\end{simple_example}
\end{baitap}

\section{Bài tập tìm kiếm nhị phân kết quả}

\begin{baitap}{Đọc sách}{https://marisaoj.com/problem/86}
Marisa có $n$ cuốn sách, cuốn thứ $i$ nằm ở vị trí $A_i$. Cô muốn lấy ra $k$ cuốn để đọc. Hãy chọn $k$ vị trí sao cho \textbf{khoảng cách nhỏ nhất giữa hai cuốn liên tiếp được lấy ra} là lớn nhất có thể, và in ra giá trị khoảng cách lớn nhất đó.

\textbf{Input}
\begin{itemize}
    \item Dòng đầu tiên gồm hai số nguyên $n, k$.
    \item Dòng thứ hai gồm $n$ số nguyên $A_i$ (không có hai vị trí trùng nhau).
\end{itemize}

\textbf{Output}
In ra một số nguyên là khoảng cách lớn nhất sao cho có thể chọn đủ $k$ cuốn sách.

\textbf{Giới hạn}
\begin{itemize}
    \item $1 \le n \le 10^5$
    \item $1 \le A_i, k \le 10^9$
\end{itemize}

\begin{simple_example}
5\ 3 & 3 \\
10\ 4\ 2\ 3\ 1 & \\
\end{simple_example}
\end{baitap}

\begin{baitap}{Giá trị lớn nhất nhỏ nhất}{https://marisaoj.com/problem/87}
Cho mảng $A$ gồm $n$ phần tử. Hãy tìm giá trị nguyên $x$ nhỏ nhất sao cho có thể chia mảng $A$ thành chính xác $k$ mảng con liên tiếp mà tổng của từng mảng con không vượt quá $x$.

\textbf{Input}
\begin{itemize}
    \item Dòng đầu tiên gồm hai số nguyên $n, k$.
    \item Dòng thứ hai gồm $n$ số nguyên $A_i$.
\end{itemize}

\textbf{Output}
In ra một số nguyên $x$ là đáp án.

\textbf{Giới hạn}
\begin{itemize}
    \item $1 \le k \le n \le 10^5$
    \item $1 \le A_i \le 10^9$
\end{itemize}

\begin{simple_example}
5\ 2 & 9 \\
5\ 4\ 3\ 2\ 1 & \\
\end{simple_example}
\end{baitap}
