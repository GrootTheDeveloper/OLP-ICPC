\chapter{KẾ HOẠCH HOẠT ĐỘNG DỰ KIẾN NĂM 2025 -- 2026}

\minitoc

\section*{Hoạt động 1: Chương trình OLP Coaching}
\begin{itemize}[leftmargin=1.2cm]
    \item \textbf{Thời gian triển khai:} Bắt đầu từ 21/12/2025 – kết thúc trước Kỳ thi OLP Tin học Sinh viên UMT lần 3 (2026) khoảng 02 tuần.
    \item \textbf{Hình thức tổ chức:} Online/Offline – tần suất 01-02 buổi/tuần.
    \item \textbf{Mục tiêu:}
    \begin{itemize}
        \item Trang bị kiến thức lập trình căn bản đến trung cấp bằng ngôn ngữ C++.
        \item Hình thành tư duy thuật toán và kỹ năng giải quyết bài toán lập trình.
        \item Hỗ trợ thành viên có khả năng tham gia và làm bài thi OLP Tin học Sinh viên UMT.
    \end{itemize}
    \item \textbf{Đối tượng tham gia:} Thành viên Ban Chuyên môn và các thành viên ban khác có nhu cầu.
    \item \textbf{Nội dung triển khai:}
    \begin{itemize}
        \item \textit{Học C++ và Thuật toán cơ bản:}
        \begin{itemize}
            \item Biên soạn giáo trình nội bộ, tổ chức bài giảng.
            \item Thực hành qua hệ thống bài tập Codeforces/VNOI/MarisaOJ/UMTOJ.
        \end{itemize}
        \item \textit{Hướng dẫn sử dụng \LaTeX:}
        \begin{itemize}
            \item Viết báo cáo tổng hợp kiến thức sau mỗi giai đoạn học.
        \end{itemize}
    \end{itemize}
\end{itemize}

\section*{Hoạt động 2: Chương trình OLP Mentoring – Đội tuyển Olympic Tin học}
\begin{itemize}[leftmargin=1.2cm]
    \item \textbf{Thời gian triển khai:} Bắt đầu sau khi kết thúc Kỳ thi OLP Tin học Sinh viên UMT lần 3 và kết thúc 01 tuần trước Kỳ thi OLP Tin học Sinh viên Toàn quốc 2026.
    \item \textbf{Hình thức tổ chức:} Online/Offline – tần suất 01 buổi/tuần.
    \item \textbf{Mục tiêu:}
    \begin{itemize}
        \item Nâng cao kiến thức về Cấu trúc dữ liệu \& Thuật toán cho đội tuyển.
        \item Rèn luyện chiến thuật thi đấu, tối ưu thời gian làm bài.
        \item Tăng khả năng cạnh tranh tại Kỳ thi OLP Tin học Sinh viên Toàn quốc 2026.
    \end{itemize}
    \item \textbf{Yêu cầu ràng buộc:}
    \begin{itemize}
        \item Ít nhất 01 thành viên APC đạt suất thi OLP Tin học Quốc gia.
    \end{itemize}
    \item \textbf{Đối tượng tham gia:}
    \begin{itemize}
        \item Thành viên đội tuyển OLP Tin học thuộc CLB APC.
        \item Các thành viên khác có năng lực phù hợp và mong muốn rèn luyện.
    \end{itemize}
\end{itemize}

\section*{Hoạt động 3: APC Contest – Cuộc thi lập trình nội bộ}
\begin{itemize}[leftmargin=1.2cm]
    \item \textbf{Thời gian tổ chức:} Triển khai định kỳ trong học kỳ Spring \& Summer.
    \item \textbf{Mục tiêu:}
    \begin{itemize}
        \item Tạo môi trường thi đấu lập trình mô phỏng Olympic Tin học (IOI format).
        \item Đào tạo đội ngũ \textit{Problem Setter}, \textit{Tester} và \textit{Contest Organizer} nội bộ.
    \end{itemize}
    \item \textbf{Nội dung triển khai:}
    \begin{itemize}
        \item \textit{Xây dựng đề thi:}
        \begin{itemize}
            \item Hướng dẫn phân tích đề, chia \textit{subtasks}, đánh giá độ khó.
        \end{itemize}
        \item \textit{Sinh test \& chấm tự động:}
        \begin{itemize}
            \item Hướng dẫn viết test generator, validator, script chấm tự động.
        \end{itemize}
        \item \textit{Quản lý hệ thống cuộc thi:}
        \begin{itemize}
            \item Sử dụng GitHub và nền tảng UMTOJ để tổ chức thử nghiệm.
        \end{itemize}
    \end{itemize}
\end{itemize}

\section*{Hoạt động 4: Hỗ trợ Khoa Công nghệ tổ chức các kỳ thi/Workshop}
\begin{itemize}[leftmargin=1.2cm]
    \item (Nội dung sẽ bổ sung sau – theo kế hoạch của Khoa Công nghệ UMT)
\end{itemize}
